\section{کارهای مرتبط پیشین}

\begin{frame}{توزیع‌بار کاری در سکو‌های پرداز بدون‌میزبان}
\begin{itemize}\RTList
	\فقره سرعت ایجاد محیط اجرا یک چالش‌ برای کارایی سکو پردازش بدون‌میزبان است.
	
	\فقره 	برای ایجاد محیط اجرا باید نیازمندی‌ها بار گیری و اجرا شوند.
%	(کتابخانه‌ها، برنامه‌های شخص ثالث و سیستم زمان اجرا)
	
	\فقره به همین منظور خدمت‌گزاران این نیازمندی‌ها را در یک حافظه نهان نگهداری می‌کنند.
\end{itemize}
	
\centering{
\mybox{کار مرتبط}{
	\مل{PASch} یک سیستم توزیع‌بار است که با توجه به نیازمندی‌های بارگیری شده روی خدمت‌گزارها
	گذردهی سیستم‌ را $1.29$ برابر افزایش  و تاخیر صدک ۸۰ام را تا ۲۳ برابر
	کاهش داده است\مرجع{aumala2019beyond}.
}}
\end{frame}


\begin{frame}{توزیع‌بار کاری در سکو‌های پرداز بدون‌میزبان}
	\begin{itemize}\RTList
		\فقره محیط‌های اجرا در حالت سرد عملکرد ضعیف‌تری نسبت به محیط گرم دارند.
		
		\فقره محیط‌های اجرا پس از اتمام کار برای مدتی باز نگه‌داشته می‌شوند.
		
		\فقره توزیع‌کننده بار می‌تواند با آگاهی از محیط‌های گرم کارایی را بهبود دهد.
	\end{itemize}
	
	\centering{
		\mybox{کار مرتبط}{
			با افزودن آگاهی از مکان محیط‌های گرم به توزیع‌کننده بار،
			نشان داده شده است که می‌توان ۶۳ درصد تاخیر در صدک ۹۹‌ام را کاهش داد\مرجع{wu2022container}.
	}}
\end{frame}


\begin{frame}{توزیع‌بار کاری در سکو‌های پرداز بدون‌میزبان}
	\begin{itemize}\RTList
		\فقره تشخیص زمان بستن یک محیط اجرا اهمیت فراوانی دارد.
		
		\فقره اگر محیط اجرا فورا بسته شود تعداد محیط‌های سرد ایجاد شده افزایش پیدا می‌کند.
		
		\فقره اگر محیط اجرا بسته نوشد هزینه خدمات بسیار زیاد می‌شود.
		
		\فقره تمام مولفه‌ها فرکانس فراخوانی یکسانی ندارند.
	\end{itemize}
	
	\centering{
		\mybox{کار مرتبط}{
			با پایش و جمع‌آوری آمار فراخوانی مولفه‌های مختلف روش‌هایی برای تعیین مدت زمان
			مناسب برای باز نگهداشتن یک محیط اجرا پیشنهاد شده است \مرجع{shahrad2020serverless}.
	}}
\end{frame}

\begin{frame}{توزیع‌بار کاری در سکو‌های پرداز بدون‌میزبان}
	
	\centering{
		\mybox{کار مرتبط}{
			برای اجرای کاربرد‌های
			بلادرنگ نیاز است تا تضمین پایان پردازش تا موعد مقرر رخداد ارائه شود.
			به همین منظور پیاده‌سازی یک الگوریتم آگاه از موعد رخدادها و توان پردازشی موجود
			نیاز است\مرجع{mampage2021deadline}.	}}
	\centering{
	\mybox{کار مرتبط}{
		تضمین کیفیت سرویس برای ارائه دهندگان خدمات ابری اهمیت دارد.
		از این منظر پیاده سازی الگوریتم‌های مختلف که
		انصاف را برای انواع بارهای کاری رعایت کند اهمیت دارد\مرجع{tariq2020sequoia}.}}

	\centering{
	\mybox{کار مرتبط}{
		ارسال پیام از یک خدمت‌گزار به یک خدمت‌گزار دیگر زمان‌بر است. از این روی طراحی سیستم
		توزیع‌بار به نحوی که زنجیره اجرا بر روی یک خدمت‌گزار قرار بگیرد باعث
		افزایش کارایی می‌شود\مرجع{jia2021nightcore}.
	}}

\end{frame}


\begin{frame}
	\begin{table}[!h]
		\centering
		\شرح{جمع‌بندی کارهای مرتبط}
		\برچسب{کارهایپیشین}
		\begin{tabular}{|c|c|c|r|}
			\hline
			مرجع & سال انتشار & استفاده از مدل تحلیلی & \multicolumn{1}{c|}{معیار مورد توجه}   \\ \hline
			\مرجع{mahmoudi2020performance}    & ۲۰۱۹       & بله                   & ارزیابی کارایی                         \\ \hline
			\مرجع{lin2020modeling}    & ۲۰۲۰       & بله                   & هزینه سرویس در مقابل کارایی            \\ \hline
			\مرجع{mampage2021deadline}   & ۲۰۲۱       & خیر                   & توجه به نیاز سیستم‌های بی درنگ         \\ \hline
			\مرجع{lee2021greedy}       & ۲۰۲۱       & خیر                   & افزایش بهره‌وری حافظه نهان             \\ \hline
			\مرجع{shahrad2020serverless}          & ۲۰۲۰       & خیر                   & تعداد محیط های اجرای سرد               \\ \hline
			\مرجع{aumala2019beyond}    & ۲۰۱۸       & خیر                   & توجه به حافظه نهان نیازمندی‌های برنامه \\ \hline
			\مرجع{wu2022container}   & ۲۰۲۲       & خیر                   & توجه به چرخه‌حیات لایه مجازی‌سازی      \\ \hline
			\مرجع{jia2021nightcore}   & ۲۰۲۱       & خیر                   & توجه به اجرای زنجیره روی یک خدمت‌گزار      \\ \hline
		\end{tabular}
	\end{table}
\end{frame}

%
%پردازش بدون‌میزبان عرصه جدیدی در پردازش ابری است که توجه پژوهشگران بسیاری
%را به خود جلب کرده است. در این بخش کارهای مرتبط را در زمینه‌های توزیع‌بار،
%ارزیابی کارایی و مدل‌سازی سکو‌های پردازش بدون‌میزبان بررسی می‌‌شود.
%
%
%\زیرقسمت{توزیع‌بار کاری در سکو‌های پرداز بدون‌میزبان}
%
%سرعت ایجاد محیط اجرا یکی از چالش‌های کارایی سیستم‌های پردازش بدون‌میزبان. برای
%ایجاد محیط اجرا با نیازمندی‌ها که شامل کتابخانه‌ها، برنامه‌های شخص ثالث و سیستم زمان اجرا
%است بار گیری و اجرا شوند. به همین منظور خدمت‌گزاران این نیازمندی‌ها را برای مولفه‌هایی که
%به تازگی اجرا کرده‌اند در یک حافظه نهان نگهداری می‌کنند تا در صورت نیاز به اجرای مجدد کارایی
%بهتری را ارائه کنند. \مل{PASch} یک سیستم توزیع‌بار است که با توجه به این موضوع توانسته است
%گذردهی این سیستم‌ها را $1.29$ برابر افزایش دهد و تاخیر صدک ۸۰ام را تا ۲۳ برابر
%کاهش دهد\مرجع{aumala2019beyond}.
%
%یکی دیگر از چالش‌هایی کارایی سیستم‌های پردازش بدون‌میزبان، عملکرد ضعیف محیط اجرا در حالت سرد است.
%به همین منظور ارائه دهندگان این خدمات، پس از پایان کار یک مولفه، محیط اجرای آن را برای مدتی فعال
%نگاه می‌دارند تا بتوان از آن استفاده مجدد کرد. متاسفانه سیستم توزیع کننده بار کاری از محیط‌های اجرای
%گرم آگاه نیست و به همین منظور نمی‌تواند از این قابلیت بهره کافی را ببرد. با افزودن این آگاهی در
%هنگام توزیع‌بار نشان داده شده است که می‌توان ۶۳ درصد تاخیر در صدک ۹۹‌ام را کاهش داد\مرجع{wu2022container}.
%
%کاربردهای بلادرنگ در دنیای پردازش از جایگاه خاصی برخوردار هستند.برای اجرای برنامه‌های
%بلادرنگ در سکو‌های پردازش بدون‌میزبان نیاز است تا تضمین پایان پردازش تا موعد مقرر رخداد
%ارائه شود. به همین منظور پیاده‌سازی یک الگوریتم آگاه از موعد رخدادها و توان پردازشی موجود
%نیاز است\مرجع{mampage2021deadline}.
%
%تضمین کیفیت سرویس برای ارائه دهندگان خدمات ابری بسیار حیاتی است. سیستم‌های توزیع‌بار 
%بهترین مکان برای اعمال سیاست‌های مختلف است. از این منظر پیاده سازی الگوریتم‌های مختلف که
%انصاف را برای انواع بارهای کاری رعایت کند اهمیت دارد\مرجع{tariq2020sequoia}.
%
%ارسال پیام از یک خدمت‌گزار به یک خدمت‌گزار دیگر زمان‌بر است. از این روی طراحی سیستم
%توزیع‌بار به نحوی که زنجیره اجرا تا جای ممکن بر روی یک خدمت‌گزار قرار بگیرد باعث
%افزایش کارایی می‌شود\مرجع{jia2021nightcore}.
%
%
%\زیرقسمت{ارزیابی کارایی سکو‌های پردازش بدون‌میزبان}
%
%ارزیابی کارایی سکو‌های پردازش بدون‌میزبان مورد توجه بسیاری قرار گرفته است.
%این مطالعات باعث می‌شود گلوگاه‌ها و نقاط ضعف سیستم‌ها مشخص شود و در نتیچه
%بتوان با طراحی بهتر سیستم‌ها کارایی بالاتری را دریافت کرد\مرجع{palade2019evaluation}.
%
%دو راهبرد برای ارزیابی کارایی سیستم‌های پردازش ابری مورد استفاده است. در روش اول 
%مطالعات مشاهداتی صورت می‌گیرد به این طریق که با اجرای بارهای کاری بر روی سیستم‌ها و
%جمع‌آوری مشاهدات و نتایج، درنهایت مقایسه‌ای میان روش‌های مختلف صورت می‌گیرد.
%در این روش بدست آوردن اطلاعات دقیق و قابل تکرار از چالش‌های موجود است. روش دیگر
%استفاده از مدل‌های تحلیلی است. طراحی این مدل‌ها دشوار است و ممکن است جزئیات مهمی
%را در هنگام ایجاد لایه‌های انتزاع از درست بدهند\مرجع{kounev2021toward}.
%
%در ارزیابی سیستم‌های بدون‌میزبان پارامتر‌های متعددی مورد توجه است که از جمله‌ی
%آن‌ها می‌توان به زمان اجرا، گذردهی، تعداد دفعات فراخوانی یک مولفه و زمان ایجاد محیط اجرا
%اشاره کرد\مرجع{shahrad2020serverless,ustiugov2021analyzing}.
%
%
%\زیرقسمت{مدل‌سازی سکو‌های پردازش بدون‌میزبان}
%
%با توجه به ذات توزیع شده سیستم‌های بدون‌میزبان و انجام آزمایش‌های
%مشاهداتی با چالش‌های بسیاری مواجه است.
%از جمله چالش تفاوت چشمگیر در روش‌های پیاده سازی سکو‌های پردازش بدون‌میزبان
%توسط ارائه دهندگان مختلف است. از این جهت مطالعات صورت گرفته برای قابل
%قیاس بودن باید بر روی چندین سکوی مختلف به صورت متناسب صورت
%بگیرد\مرجع{palade2019evaluation, mcgrath2017serverless}.
%از طرف دیگر، بارهای کاری پردازش بدون‌میزبان متفاوت است و مطالعات مشاهداتی
%فقط برای یک حوزه می‌تواند کاربردی باشد.
%در صورت انجام چنین مشاهداتی، اطلاعات بدست آمده در کاربرد مورد مطالعه بسیار
%ارزشمند است\مرجع{shahrad2020serverless}.
%
%
%با توجه به پیچیدگی‌های انجام آزمایش‌های مشاهداتی، انجام شبیه‌سازی‌ جنبه‌های
%مختلف سکوی ارائه خدمات بسیار مورد توجه قرار گرفته شده
%است\مرجع{shahrad2020serverless,mahmoudi2021simfaas,jeon2019cloudsim}.
%معمولا شبیه‌سازی‌ها بر روی ویژگی‌های محدودی از معیارهای کارایی تمرکز می کنند.
%مواردی مانند تعداد محیط‌های سرد ایجاد شده یا تعداد پیام‌های از دست رفته از جمله
%این معیارها است.
%
%استفاده از روش‌های تحلیلی باعث فهم ویژگی‌های ذاتی موضوع مورد مطالعه می‌شوند. این
%روش‌ها امکان مقایسه پیاده‌سازی‌های عملی را با حدبالای تئوری قابل دست‌یابی می‌دهند و به
%طراحان کمک می‌کنند که درباره جنبه‌های مختلف سیستم تصمیم بگیرند.
%از طرفی بدست‌آوردن مدل‌های تحلیلی همیشه ممکن نیست.
%در کار‌های گذشته تلاش‌هایی برای مدل‌سازی سکو‌های پردازش بدون‌میزبان ارائه 
%شده است\مرجع{mahmoudi2020performance,lin2020modeling,mampage2021deadline,lee2021greedy}.
%مدل‌های بدست آمده امکان بهینه‌سازی چندین ويزگی حائز اهمیت مانند، کارایی و هزینه
%مورد نیاز سیستم را فراهم می‌آورد.
%
%در جدول \رجوع{کارهایپیشین} خلاصه‌ای از رویکرد کارهای پیشین به بررسی این مسئله پرداخته شده است.
%
%\begin{table}[!h]
%	\centering
%	\شرح{جمع‌بندی کارهای مرتبط}
%	\برچسب{کارهایپیشین}
%	\begin{tabular}{|c|c|c|r|}
%		\hline
%		مرجع & سال انتشار & استفاده از مدل تحلیلی & \multicolumn{1}{c|}{معیار مورد توجه}   \\ \hline
%		\مرجع{mahmoudi2020performance}    & ۲۰۱۹       & بله                   & ارزیابی کارایی                         \\ \hline
%		\مرجع{lin2020modeling}    & ۲۰۲۰       & بله                   & هزینه سرویس در مقابل کارایی            \\ \hline
%		\مرجع{mampage2021deadline}   & ۲۰۲۱       & خیر                   & توجه به نیاز سیستم‌های بی درنگ         \\ \hline
%		\مرجع{lee2021greedy}       & ۲۰۲۱       & خیر                   & افزایش بهره‌وری حافظه نهان             \\ \hline
%		\مرجع{shahrad2020serverless}          & ۲۰۲۰       & خیر                   & تعداد محیط های اجرای سرد               \\ \hline
%		\مرجع{aumala2019beyond}    & ۲۰۱۸       & خیر                   & توجه به حافظه نهان نیازمندی‌های برنامه \\ \hline
%		\مرجع{wu2022container}   & ۲۰۲۲       & خیر                   & توجه به چرخه‌حیات لایه مجازی‌سازی      \\ \hline
%		\مرجع{jia2021nightcore}   & ۲۰۲۱       & خیر                   & توجه به اجرای زنجیره روی یک خدمت‌گزار      \\ \hline
%	\end{tabular}
%\end{table}

