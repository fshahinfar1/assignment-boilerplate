\قسمت*{چکیده}
\markboth{چکیده}{}

پردازش بدون‌میزبان تحولی در پردازش ابری است و فرصت‌ها و چالش‌های فراوانی
را به وجود آورده است. مقیاس‌پذیری و اتکاپذیری رکن اساسی این رویکرد است که
از طریق زیر سیستم همنوازی موجود در سکو‌های ارائه خدمات به تحقق می‌پیوندد.
توزیع‌بار بخشی از زیر سیستم همنوازی است که امکان اعمال سیاست‌های مورد نظر را
برای انتخاب زمان پردازش درخواست و خدمت‌گزار مناسب فراهم می‌آورد.
از این روی فهم نحوه عملکرد سیاست‌های توزیع‌بار اهمیت دارد. در این پروژه با استفاده
از روش‌های مدل‌سازی و شبیه‌سازی اقدام به بررسی الگوریتم‌های توزیع‌بار می‌گردد
و با توجه به نتایج مقایسات یک الگوریتم توزیع‌بار جدید ارائه می‌شود.

\متن‌سیاه{واژه‌های کلیدی:}
‫پردازش بدون‌میزبان، پردازش ابری، توزیع با، شبکه‌های مراکز داده، مدل سازی کارایی سیستم‌های کامپیوتری