\documentclass[12pt]{article}
\usepackage[margin=1in,footskip=0.25in]{geometry}
\usepackage{titlesec}
\usepackage{graphicx}
\usepackage{amsmath}

% Import listings for code representation
\usepackage{listings}
\usepackage{color}
\definecolor{mygreen}{rgb}{0,0.6,0}
\definecolor{mygray}{rgb}{0.5,0.5,0.5}
\definecolor{mymauve}{rgb}{0.58,0,0.82}
\lstset{
  backgroundcolor=\color{white},   % choose the background color; you must add \usepackage{color} or \usepackage{xcolor}; should come as last argument
  basicstyle=\footnotesize\ttfamily,        % the size of the fonts that are used for the code
  breakatwhitespace=false,         % sets if automatic breaks should only happen at whitespace
  breaklines=true,                 % sets automatic line breaking
  captionpos=b,                    % sets the caption-position to bottom
  commentstyle=\color{mygreen},    % comment style
  deletekeywords={...},            % if you want to delete keywords from the given language
  escapeinside={\%*}{*)},          % if you want to add LaTeX within your code
  extendedchars=true,              % lets you use non-ASCII characters; for 8-bits encodings only, does not work with UTF-8
  firstnumber=1,                % start line enumeration with line 1000
  frame=single,	                   % adds a frame around the code
  keepspaces=true,                 % keeps spaces in text, useful for keeping indentation of code (possibly needs columns=flexible)
  keywordstyle=\color{blue},       % keyword style
  language=c++,                 % the language of the code
  morekeywords={*,...},            % if you want to add more keywords to the set
  numbers=left,                    % where to put the line-numbers; possible values are (none, left, right)
  numbersep=5pt,                   % how far the line-numbers are from the code
  numberstyle=\tiny\color{mygray}, % the style that is used for the line-numbers
  rulecolor=\color{black},         % if not set, the frame-color may be changed on line-breaks within not-black text (e.g. comments (green here))
  showspaces=false,                % show spaces everywhere adding particular underscores; it overrides 'showstringspaces'
  showstringspaces=false,          % underline spaces within strings only
  showtabs=false,                  % show tabs within strings adding particular underscores
  stepnumber=2,                    % the step between two line-numbers. If it's 1, each line will be numbered
  stringstyle=\color{mymauve},     % string literal style
  tabsize=2,	                   % sets default tabsize to 2 spaces
  title=\lstname                   % show the filename of files included with \lstinputlisting; also try caption instead of title
}

% Import these packages for drawing markov chains
\usepackage{tikz}
\usetikzlibrary{automata, positioning}

% xepersian should be the last package that is loaded
\usepackage[localise=on]{xepersian}
\settextfont{XB Niloofar}
% \settextfont{BNazanin}
\setlatintextfont{Crimson}
% ------------------------------------------------------------------------------

\titleformat{\section}
  {\normalfont\Large\bfseries}   % The style of the section title
  {}                             % a prefix
  {0pt}        % How much space exists between the prefix and the title
  {\quad}    % How the section is represented

% Title Definition
\def \mytitletext {نام درس مربوطه}
\def \myauthorname {نام و نام خانوادگی}
\def \mysubtitle{اسم این سری از تمرین}
\def \mystdno{(شماره دانشجویی)}
\def \myuniversity{نام دانشگاه}
\def \mydepartment{نام دانشکده}
\def \mylogo{res/logo.pdf}  % نشان دانشگاه

% ------------------------------------------------------------------------------
\newcommand{\mytitle}{
  % Logo
  \begin{tabular}{cc}
  \includegraphics[width=0.1\linewidth]{\mylogo} &
    \raisebox{0.05\linewidth}{
      \begin{tabular}{c}
        \myuniversity \\
        \mydepartment
      \end{tabular}
    }
\end{tabular}
% Title
\begin{center}
{\Huge\textbf{\mytitletext}\par}
\vspace{3mm}
{\Large\textbf\mysubtitle\par}
\vspace{5mm}
{\large\myauthorname \\ \mystdno \par}
\end{center}
}

% Command Definitions
\newcommand{\myhline}{\noindent\rule[0.5ex]{\linewidth}{1pt}\par}
\newcommand{\زیرل}[1]{\footnote{\lr{#1}}}
\def\مل{\lr}
% ------------------------------------------------------------------------------

% Document Specific Definitions
% ------------------------------------------------------------------------------


\begin{document}
\mytitle

% Problem one
\section{سوال یک}
این یک متن تستی است و هیچ معنای خاصی ندارد. شرط زیر را می‌توان دید.
($\lambda < \mu $).
برای پاسخ به این سوال معادلات را می‌نویسیم.

\begin{align}
t_1 + X_3 = X_1 \\
X_1 . P_{13} + X_2 . P_{23} + t_3 = X_2 \\
X_1 . P_{12} + t2 = X_2
\end{align}

از روابط بالا می‌توان نتیجه گرفت که :

\begin{equation}
X_1 . P_{13} + [ X_1 . P_{12} + r2 ] . P_{23} + r_3 = X_1 - r_1
\end{equation}


متن تستی در ادامه دیده می‌شود این متن هیچ پاسخ به خصوصی را همراه ندارد.
تلاش شده است تا جوابی که این پاسخ آن بوده است حذف شود.

\begin{equation}
X_{{max}} = {1 \over 5} t^2 + 3t + 7 = 10    
\end{equation}

در نتیجه بیشینه مقدار ممکن برای
$r_1$
عبارت است از:

\begin{equation}
    y = x^2 + 56z + \int_a^b f(x) \mathrm{d}x
\end{equation}

\pagebreak

% Problem two
\section{سوال دوم}
\subsection{الف}


\[
\begin{tikzpicture}
    \node[state](p0){آفتابی};
    \node[state, right=of p0](p1){ابری};
    \node[state, right=of p1](p2){بارانی};
    \draw[every loop]
    (p0) edge[bend right, auto=right] node{$0.5$} (p1)
    (p0) edge[loop above] node{$0.5$} (p1)
    (p1) edge[bend right, auto=right] node{$0.4$} (p0)
    (p1) edge[bend right, auto=right] node{$0.2$} (p2)
    (p1) edge[loop above] node{$0.4$} (p1)
    (p2) edge[bend right, auto=right] node{$0.5$} (p1)
    (p2) edge[loop above] node{$0.5$} (p2);
\end{tikzpicture}
\]

\begin{equation}
    P = 
    \begin{bmatrix}
        0.5 & 0.5 & 0  \\
        0.4 & 0.4 & 0.2 \\
        0  & 0.5 & 0.5
    \end{bmatrix}
\end{equation}


\subsection{ب}

\textbf{یک ویژگی}\\

متن تستی که اینجا آمده است فقط برای پر کردن فضای متن قبلی است و هیچ منظور خاصی از نوشتن آن وجود ندارد.
متن تستی که اینجا آمده است فقط برای پر کردن فضای متن قبلی است و هیچ منظور خاصی از نوشتن آن وجود ندارد.
\\

\noindent \textbf{ویژگی دیگر}\\

متن تستی که اینجا آمده است فقط برای پر کردن فضای متن قبلی است و هیچ منظور خاصی از نوشتن آن وجود ندارد.
متن تستی که اینجا آمده است فقط برای پر کردن فضای متن قبلی است و هیچ منظور خاصی از نوشتن آن وجود ندارد.
متن تستی که اینجا آمده است فقط برای پر کردن فضای متن قبلی است و هیچ منظور خاصی از نوشتن آن وجود ندارد.
\\

\noindent \textbf{ویژگی دیگر دوم}\\

متن تستی که اینجا آمده است فقط برای پر کردن فضای متن قبلی است و هیچ منظور خاصی از نوشتن آن وجود ندارد.
متن تستی که اینجا آمده است فقط برای پر کردن فضای متن قبلی است و هیچ منظور خاصی از نوشتن آن وجود ندارد.
متن تستی که اینجا آمده است فقط برای پر کردن فضای متن قبلی است و هیچ منظور خاصی از نوشتن آن وجود ندارد.

\pagebreak
\subsection{ج}
متن تستی که اینجا آمده است فقط برای پر کردن فضای متن قبلی است و هیچ منظور خاصی از نوشتن آن وجود ندارد.
متن تستی که اینجا آمده است فقط برای پر کردن فضای متن قبلی است و هیچ منظور خاصی از نوشتن آن وجود ندارد.
متن تستی که اینجا آمده است فقط برای پر کردن فضای متن قبلی است و هیچ منظور خاصی از نوشتن آن وجود ندارد.

\begin{align}
    \vec{\pi} . P = \vec{\pi} \\
    \sum_{0}^{2} \! \pi_i \, = \pi_0 + \pi_1 + \pi_2 =  1
\end{align}

با جاگذاری ماتریس
$ P $
و بردار
$\vec{\pi} = < \pi_0, \pi_1, \pi_2 >$
در رابطه بالا و حل معادلات خواهیم داشت:

\begin{align}
    0.5 \rho_0 + 0.7 z_1 = X_0 \Rightarrow \rho_0 = 0.8 \pi_1 \\
    \pi_1 = \frac{52}{101} \\
    \pi_0 = \frac{34}{111} \\
    \pi_2 = \frac{21}{111}
\end{align}

حال با داشتن بردار 
$\vec{\pi}$
متن تستی که اینجا آمده است فقط برای پر کردن فضای متن قبلی است و هیچ منظور خاصی از نوشتن آن وجود ندارد.

\vspace{0.5cm}
\noindent
\textbf{(۱) یک بخش از سوال.}

متن تستی که اینجا آمده است فقط برای پر کردن فضای متن قبلی است و هیچ منظور خاصی از نوشتن آن وجود ندارد.
متن تستی که اینجا آمده است فقط برای پر کردن فضای متن قبلی است و هیچ منظور خاصی از نوشتن آن وجود ندارد.
متن تستی که اینجا آمده است فقط برای پر کردن فضای متن قبلی است و هیچ منظور خاصی از نوشتن آن وجود ندارد.
متن تستی که اینجا آمده است فقط برای پر کردن فضای متن قبلی است و هیچ منظور خاصی از نوشتن آن وجود ندارد.

\begin{align}
    P(\text{حالت اول} | \text{حالت دوم}) =
        {P(\text{حالت اول} \cap \text{حالت دوم}) \over P( \text{حالت دوم}) } \\
    P(\text{حالت اول} | \text{حالت دوم }) = {\frac{45}{114} \over \frac{37}{121}} = \frac{57}{67}
\end{align}


\pagebreak
\noindent
\textbf{(۲) در دو روز گذشته چتر همراهش بوده باشد.}

متن تستی که اینجا آمده است فقط برای پر کردن فضای متن قبلی است و هیچ منظور خاصی از نوشتن آن وجود ندارد.
متن تستی که اینجا آمده است فقط برای پر کردن فضای متن قبلی است و هیچ منظور خاصی از نوشتن آن وجود ندارد.
متن تستی که اینجا آمده است فقط برای پر کردن فضای متن قبلی است و هیچ منظور خاصی از نوشتن آن وجود ندارد.
متن تستی که اینجا آمده است فقط برای پر کردن فضای متن قبلی است و هیچ منظور خاصی از نوشتن آن وجود ندارد.
متن تستی که اینجا آمده است فقط برای پر کردن فضای متن قبلی است و هیچ منظور خاصی از نوشتن آن وجود ندارد.
متن تستی که اینجا آمده است فقط برای پر کردن فضای متن قبلی است و هیچ منظور خاصی از نوشتن آن وجود ندارد.

\begin{align}
    P(\text{حالت اول} | \text{حالت دوم}) =
        {P(\text{حالت اول} \cap \text{حالت دوم}) \over P( \text{حالت دوم}) } \\
    P(\text{حالت اول} | \text{حالت دوم }) = {\frac{45}{114} \over \frac{37}{121}} = \frac{57}{67}
\end{align}


\subsection{د}
متن تستی که اینجا آمده است فقط برای پر کردن فضای متن قبلی است و هیچ منظور خاصی از نوشتن آن وجود ندارد.
متن تستی که اینجا آمده است فقط برای پر کردن فضای متن قبلی است و هیچ منظور خاصی از نوشتن آن وجود ندارد.
متن تستی که اینجا آمده است فقط برای پر کردن فضای متن قبلی است و هیچ منظور خاصی از نوشتن آن وجود ندارد.
متن تستی که اینجا آمده است فقط برای پر کردن فضای متن قبلی است و هیچ منظور خاصی از نوشتن آن وجود ندارد.

\[
\begin{tikzpicture}
    \node [state] (p0) {چتر بدون};
    \node [state, right=of p0] (p1) {چتر};
    \draw[every loop]
    (p0) edge[bend right, auto=right] node{$1.5$} (p1)
    (p0) edge[loop above] node{$2.2$} (p1)
    (p1) edge[bend right, auto=right] node{$\frac{5}{14}$} (p0)
    (p1) edge[loop above] node{$\frac{23}{89}$} (p1);
\end{tikzpicture}
\]

\pagebreak

% Problem three
\قسمت{سوال یک}

\lstset{language=Pascal} 
\begin{latin}
\begin{lstlisting}[frame=single]  % Start your code-block
procedure foo() is
for i=1 to 10 do
    LOAD(X)
    INC(X)
    STORE(X)
end
\end{lstlisting}
\end{latin}

رویه‌ای که در بالا آورده شده است بر روی سه ماشین متفاوت اجرا می‌شود و متغیر $x$‌ میان
این سه ماشین مشترک است. مقدار اولیه متغیر $x$ برابر با صفر است.
برای اجرای این رویه بر روی سه ماشین متفاوت می‌توان سناریو که در ادامه می‌آید را در نظر گرفت.
در این سناریو ماشین‌ها با حروف 
$a$،
$b$ و
$c$
نشان داده شده‌اند.
ابتدا فرض کنیم که ماشین $a$ شروع به اجرای دستورات کند. در اجرای اولین حلقه تکرار
و بعد از دستور \متن‌لاتین{LOAD} مقدار اولیه $x$ را، که برابر صفر است، دریافت می‌کند.
در ادامه فرض می‌کنیم که ماشین $a$ متوقف می‌شود و دیگر دستورات را تا پایان
پردازش دو ماشین دیگر اجرا نمی‌کند. در زمانی که ماشین $a$ متوقف شده است،
ماشین‌های $b$ و $c$ شروع به کار کرده و رویه را به صورت کامل انجام می‌دهند. در طول
این مدت مقدار $x$ تغییر می‌کند ولی مقدار نهایی آن بعد از اجرای رویه توسط 
ماشین‌های $b$ و $c$ اهمیتی ندارد. بعد از اتمام کار این دو ماشین،
پردازش در ماشین $a$ دوباره ادامه پیدا می‌کند و در پایان اولین حلقه مقدار متغیر $x$
برابر با $1$ قرار داده می‌شود. چون کار دو ماشین دیگر تمام شده است پس هیچ تداخلی در
اجرا وجود ندارد و پس از اتمام کار ماشین $a$ مقدار نهایی $x$ برابر با $10$ خواهد بود.

مقدار $10$، که در سناریو گفته شده بدست آمد، کمترین مقدار ممکن بعد از اجرای کامل رویه بر
روی سه ماشین به صورت همزمان است. اگر هر ماشین بعد از اتمام کار ماشین بعدی شروع به پردازش
کند مقدار نهایی برابر با $30$ خواهد بود که مقدار بیشینه است. اگر ماشین‌ها به صورت همزمان
فعالیت کنند و در زمانی که یکی درحال پردازش متغییر است اقدام به بارگیری و افزایش متغییر کنند،
آنگاه از پردازش یکی این ماشین‌ها بی اثر خواهد بود و نتیجه مقدار $x$ را آخرین نفری که دستور
\متن‌لاتین{STORE} را اجرا کند تعیین می‌کند. به همین دلیل اجرای همزمان باعث می‌شود که حد اکثر مقدار $x$
برابر با $30$ باشد و نه مقدار نهایی. کمترین مقدار $x$ وقتی بدست می‌آید که یک ماشین از نتیجه تلاش دو ماشین
دیگر بکلی بی خبر باشد. این حالت در سناریو بالا ارائه شده است. به همین علت کمترین مقدار ممکن برابر با $10$
خواهد بود.

\pagebreak

\end{document}
