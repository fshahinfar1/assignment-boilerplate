\section{سوال دوم}
\subsection{الف}


\[
\begin{tikzpicture}
    \node[state](p0){آفتابی};
    \node[state, right=of p0](p1){ابری};
    \node[state, right=of p1](p2){بارانی};
    \draw[every loop]
    (p0) edge[bend right, auto=right] node{$0.5$} (p1)
    (p0) edge[loop above] node{$0.5$} (p1)
    (p1) edge[bend right, auto=right] node{$0.4$} (p0)
    (p1) edge[bend right, auto=right] node{$0.2$} (p2)
    (p1) edge[loop above] node{$0.4$} (p1)
    (p2) edge[bend right, auto=right] node{$0.5$} (p1)
    (p2) edge[loop above] node{$0.5$} (p2);
\end{tikzpicture}
\]

\begin{equation}
    P = 
    \begin{bmatrix}
        0.5 & 0.5 & 0  \\
        0.4 & 0.4 & 0.2 \\
        0  & 0.5 & 0.5
    \end{bmatrix}
\end{equation}


\subsection{ب}

\textbf{یک ویژگی}\\

متن تستی که اینجا آمده است فقط برای پر کردن فضای متن قبلی است و هیچ منظور خاصی از نوشتن آن وجود ندارد.
متن تستی که اینجا آمده است فقط برای پر کردن فضای متن قبلی است و هیچ منظور خاصی از نوشتن آن وجود ندارد.
\\

\noindent \textbf{ویژگی دیگر}\\

متن تستی که اینجا آمده است فقط برای پر کردن فضای متن قبلی است و هیچ منظور خاصی از نوشتن آن وجود ندارد.
متن تستی که اینجا آمده است فقط برای پر کردن فضای متن قبلی است و هیچ منظور خاصی از نوشتن آن وجود ندارد.
متن تستی که اینجا آمده است فقط برای پر کردن فضای متن قبلی است و هیچ منظور خاصی از نوشتن آن وجود ندارد.
\\

\noindent \textbf{ویژگی دیگر دوم}\\

متن تستی که اینجا آمده است فقط برای پر کردن فضای متن قبلی است و هیچ منظور خاصی از نوشتن آن وجود ندارد.
متن تستی که اینجا آمده است فقط برای پر کردن فضای متن قبلی است و هیچ منظور خاصی از نوشتن آن وجود ندارد.
متن تستی که اینجا آمده است فقط برای پر کردن فضای متن قبلی است و هیچ منظور خاصی از نوشتن آن وجود ندارد.

\pagebreak
\subsection{ج}
متن تستی که اینجا آمده است فقط برای پر کردن فضای متن قبلی است و هیچ منظور خاصی از نوشتن آن وجود ندارد.
متن تستی که اینجا آمده است فقط برای پر کردن فضای متن قبلی است و هیچ منظور خاصی از نوشتن آن وجود ندارد.
متن تستی که اینجا آمده است فقط برای پر کردن فضای متن قبلی است و هیچ منظور خاصی از نوشتن آن وجود ندارد.

\begin{align}
    \vec{\pi} . P = \vec{\pi} \\
    \sum_{0}^{2} \! \pi_i \, = \pi_0 + \pi_1 + \pi_2 =  1
\end{align}

با جاگذاری ماتریس
$ P $
و بردار
$\vec{\pi} = < \pi_0, \pi_1, \pi_2 >$
در رابطه بالا و حل معادلات خواهیم داشت:

\begin{align}
    0.5 \rho_0 + 0.7 z_1 = X_0 \Rightarrow \rho_0 = 0.8 \pi_1 \\
    \pi_1 = \frac{52}{101} \\
    \pi_0 = \frac{34}{111} \\
    \pi_2 = \frac{21}{111}
\end{align}

حال با داشتن بردار 
$\vec{\pi}$
متن تستی که اینجا آمده است فقط برای پر کردن فضای متن قبلی است و هیچ منظور خاصی از نوشتن آن وجود ندارد.

\vspace{0.5cm}
\noindent
\textbf{(۱) یک بخش از سوال.}

متن تستی که اینجا آمده است فقط برای پر کردن فضای متن قبلی است و هیچ منظور خاصی از نوشتن آن وجود ندارد.
متن تستی که اینجا آمده است فقط برای پر کردن فضای متن قبلی است و هیچ منظور خاصی از نوشتن آن وجود ندارد.
متن تستی که اینجا آمده است فقط برای پر کردن فضای متن قبلی است و هیچ منظور خاصی از نوشتن آن وجود ندارد.
متن تستی که اینجا آمده است فقط برای پر کردن فضای متن قبلی است و هیچ منظور خاصی از نوشتن آن وجود ندارد.

\begin{align}
    P(\text{حالت اول} | \text{حالت دوم}) =
        {P(\text{حالت اول} \cap \text{حالت دوم}) \over P( \text{حالت دوم}) } \\
    P(\text{حالت اول} | \text{حالت دوم }) = {\frac{45}{114} \over \frac{37}{121}} = \frac{57}{67}
\end{align}


\pagebreak
\noindent
\textbf{(۲) در دو روز گذشته چتر همراهش بوده باشد.}

متن تستی که اینجا آمده است فقط برای پر کردن فضای متن قبلی است و هیچ منظور خاصی از نوشتن آن وجود ندارد.
متن تستی که اینجا آمده است فقط برای پر کردن فضای متن قبلی است و هیچ منظور خاصی از نوشتن آن وجود ندارد.
متن تستی که اینجا آمده است فقط برای پر کردن فضای متن قبلی است و هیچ منظور خاصی از نوشتن آن وجود ندارد.
متن تستی که اینجا آمده است فقط برای پر کردن فضای متن قبلی است و هیچ منظور خاصی از نوشتن آن وجود ندارد.
متن تستی که اینجا آمده است فقط برای پر کردن فضای متن قبلی است و هیچ منظور خاصی از نوشتن آن وجود ندارد.
متن تستی که اینجا آمده است فقط برای پر کردن فضای متن قبلی است و هیچ منظور خاصی از نوشتن آن وجود ندارد.

\begin{align}
    P(\text{حالت اول} | \text{حالت دوم}) =
        {P(\text{حالت اول} \cap \text{حالت دوم}) \over P( \text{حالت دوم}) } \\
    P(\text{حالت اول} | \text{حالت دوم }) = {\frac{45}{114} \over \frac{37}{121}} = \frac{57}{67}
\end{align}


\subsection{د}
متن تستی که اینجا آمده است فقط برای پر کردن فضای متن قبلی است و هیچ منظور خاصی از نوشتن آن وجود ندارد.
متن تستی که اینجا آمده است فقط برای پر کردن فضای متن قبلی است و هیچ منظور خاصی از نوشتن آن وجود ندارد.
متن تستی که اینجا آمده است فقط برای پر کردن فضای متن قبلی است و هیچ منظور خاصی از نوشتن آن وجود ندارد.
متن تستی که اینجا آمده است فقط برای پر کردن فضای متن قبلی است و هیچ منظور خاصی از نوشتن آن وجود ندارد.

\[
\begin{tikzpicture}
    \node [state] (p0) {چتر بدون};
    \node [state, right=of p0] (p1) {چتر};
    \draw[every loop]
    (p0) edge[bend right, auto=right] node{$1.5$} (p1)
    (p0) edge[loop above] node{$2.2$} (p1)
    (p1) edge[bend right, auto=right] node{$\frac{5}{14}$} (p0)
    (p1) edge[loop above] node{$\frac{23}{89}$} (p1);
\end{tikzpicture}
\]
