\documentclass[sigconf]{acmart}

\input{includes}

\setcopyright{none}
\settopmatter{printacmref=false, printccs=false, printfolios=false}

\copyrightyear{2022}
\acmYear{2022}
% \setcopyright{rightsretained}
\acmConference[]{}
\acmBooktitle{}
\acmDOI{}
\acmISBN{}


\begin{document}

%%
%% The "title" command has an optional parameter,
%% allowing the author to define a "short title" to be used in page headers.
\title{A new abstraction for programming network functions}

\author{FirstName LastName}
\affiliation{
	\institution{University}
	\country{}
}
\email{email@mailserver.edu}
\orcid{1234-5678-9012}

\author{FirstName LastName}
\affiliation{
	\institution{University}
	\country{}
}
\email{email@mailserver.edu}
\orcid{1234-5678-9012}


%%
%% By default, the full list of authors will be used in the page
%% headers. Often, this list is too long, and will overlap
%% other information printed in the page headers. This command allows
%% the author to define a more concise list
%% of authors' names for this purpose.
% \renewcommand{\shortauthors}{author et al.}

\begin{abstract}
  A clear and well-documented \LaTeX\ document is presented as an
  article formatted for publication by ACM in a conference proceedings
  or journal publication. Based on the ``acmart'' document class, this
  article presents and explains many of the common variations, as well
  as many of the formatting elements an author may use in the
  preparation of the documentation of their work.
\end{abstract}

\keywords{datasets, neural networks, gaze detection, text tagging}

\maketitle

\input{sections/introduction.tex}

%% Uncommnet for adding acknowledgement section
% \begin{acks}
% ACKNOWLEDGEMENT is placed here!
% \end{acks}

\bibliographystyle{ACM-Reference-Format}
\bibliography{refs}

% \input{sections/appendix}

\end{document}
\endinput
%%
%% End of file `sample-sigconf.tex'.
