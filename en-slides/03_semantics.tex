\begin{frame}
    \frametitle{Translation}
    \begin{itemize}
        \item {Petri-Nets are defined as a tuple shown below}
    \end{itemize}
    \[C = (P, T, I, O, \mu)\]
    \begin{itemize}
        \item {In the rest of this section we define Petri-Nets with transition systems defined as:}
    \end{itemize}
    \begin{gather*}
    (S, Act, \xrightarrow{a}, Init, AP, L)
    \end{gather*}
\end{frame}

\begin{frame}
    \begin{itemize}
        \item {State of the system is tuple with the size equal to the number of places in the system}
    \end{itemize}
    \begin{gather*}
        (S, Act, \xrightarrow{a}, Init, AP, L) \\
        \mathbb{W} = \{0, 1, 2, 3, ...\} ~ ~ ; ~ n = | P |
        \\
        S =\mathbb{W} ^ n
    \end{gather*}
\end{frame}

\begin{frame}
    \begin{itemize}
        \item {For each transition $t_j$ in the Petri-Net there exists
        an action $a_j$
        having $I_{a_j}$ and $O_{a_j}$.}
        \item {$I_{a_j}$ is a sequence of zeros and ones. If $p_i$ is input of $t_j$ then $I_{a_j}[i] = 1$}
        \item {$O_{a_j}$ is a sequence of zeros and ones. If $p_i$ is output of $t_j$ then $O_{a_j}[i] = 1$}
    \end{itemize}
    \begin{gather*}
        (S, Act, \xrightarrow{a}, Init, AP, L) \\
        \mathbb{W} = \{0, 1, 2, 3, ...\} ~ ~ ; ~ n = | P |
        \\
        S =\mathbb{W} ^ n
        \\
        Act = \{a_j | ~ t_j \in T\} ~ ~ ; a_j = < I_{a_j},  O_{a_j} ~ >
        \\
        I_{a_j} = < ~ val(p_i \in I(t_j)) ~ | p_i \in P>
        \\
        O_{a_j} = < ~ val(p_i \in O(t_j)) ~ | p_i \in P>
    \end{gather*}
\end{frame}

\begin{frame}
    \frametitle{Translation}
    % \begin{itemize}
    % \item {Petri-Nets are defined as a tuple shown below}
    % \end{itemize}
    \begin{gather*}
        \mathbb{W} = \{0, 1, 2, 3, ...\} ~ ~ ; n = | P |
        \\
        S =\mathbb{W} ^ n
        \\
        Act = \{a_j   | ~ t_j \in T\} ~ ~ ; a_j = < I_{a_j},  O_{a_j} ~ >
        \\
        I_{a_j} = < ~ val(p_i \in I(t_j)) ~ | p_i \in P>
        \\
        O_{a_j} = < ~ val(p_i \in O(t_j)) ~ | p_i \in P>
        \\
        Init \in S ~ ~; Init = <~ \mu^0(p_i), ...  ~ |  p_i \in P ~ >
        \\
        AP = S
        \\
        L(s_i) = {s_i,}
    \end{gather*}
\end{frame}

\begin{frame}
    \frametitle{Translation}
    \begin{itemize}
        \item {Transition semantic can be defined as below}
    \end{itemize}
    \begin{gather*}
        \xrightarrow{a_j} \ = S \times Act \times S
        \\
        {
        I_{a_j}[k] > 0 \implies s_i[k] > 0
        \over
        s_i \xrightarrow{a_j} s'_i
        }
        \\
        {
        s_i \xrightarrow{a_j} s'_i
        \over
        s'_j = ~ < ~ s_i[k] + O_{a_j}[k] - I_{a_j}[k] , ... ~ >
        }
    \end{gather*}
\end{frame}


\begin{frame}
    \frametitle{Example}
    % \begin{itemize}
    %     \item {Transition semantic can be defined as below}
    % \end{itemize}

    % \tikzset{node distance = 0.5cm and 0.5cm}
    \begin{figure}
        \centering
        \intropetrinet[black][black][0.6]{}
    \end{figure}
\end{frame}
\begin{frame}
    \frametitle{Example}
    \begin{itemize}
        \item {Transition semantic can be defined as below}
    \end{itemize}
    {\small
    \begin{gather*}
        S = \{(0,0,0,0,0), (1,0,0,0,0), ...,(1,1,1,0,0),  ..., (1,0,1,0,2), ...\}
        \\
        Act = \{a_1=<(1,0,0,0,0), (0,1,1,0,1)>, a_2=<(0, 1, 1, 0, 1), (0,0,0,0,1)>,
        \\ a_3=<(0,0,1,0,0), (0,0,0,1,0)>,  a_4=<(0,0,0,-1,0),(0,1,1,0,0)>\}
        \\
        \xrightarrow{a_j} \ = S \times Act \times S
        \\
        Init = (1,0,1,0,2)
        \\
        AP = S
        \\
        L(s_i) = {s_i}
    \end{gather*}
    }
\end{frame}

\begin{frame}
\frametitle{Example Execution}
\begin{figure}[!h]
\small
\centering
\begin{tikzpicture}[thick,scale=0.6, every node/.style={scale=0.6}]
\node[state,blue](s0){(1,0,1,0,2)};
\node[state, below left=of s0,red,xshift=-2cm](s1){(0,1,2,0,3)};
\node[state, below right=of s0,xshift=2cm](s2){(1,0,0,1,2)};
\node[state,below left=of s1](s3){(0,0,1,0,3)};
\node[state,below right=of s1 ](s4){(0,1,1,1,3)};
\node[below=of s2](emptys2){$\vdots$};
\node[below=of s3](emptys3){$\vdots$};
\node[below=of s4](emptys4){$\vdots$};
\draw[every loop,thick]
(s0) edge[bend right, auto=right] node{$a_1$} (s1)
(s0) edge[bend left, auto=left] node{$a_3$} (s2)
(s1) edge[bend right, auto=right] node{$a_2$} (s3)
(s1)edge[bend left, auto=left] node{$a_3$} (s4)
;
\end{tikzpicture}
\end{figure}
\end{frame}
