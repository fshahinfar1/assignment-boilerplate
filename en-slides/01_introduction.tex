\begin{frame}
    \frametitle{What is a Petri-Net}
    \begin{itemize}
        \item It is a modeling language.
        \item It is a \emph{directed} \emph{bipartite} graph.
        \item Petri-Nets have two type of nodes: \emph{places} and \emph{transitions}.
        \item Arcs connect \emph{places} to \emph{transitions} and vice-versa.
        \item Places hold \emph{tokens}.
        \item \emph{Tokens} determine the state (also called marking) of the system.
    \end{itemize}
\end{frame}

\begin{frame}
    \frametitle{Example of a Petri-Net graph}
    \begin{itemize}
        \item{Circles are \emph{places}}
        \item{Rectangles are \emph{transitions}}
        \item{Dots are \emph{tokens}}
    \end{itemize}
    \begin{figure}
        \centering
        \intropetrinet{}
    \end{figure}
\end{frame}

\begin{frame}
    \frametitle{Transition Rules}
    \begin{itemize}
        \item{A transition is \emph{enabled} if all its input \emph{places} has at-least one token.}
        \item{If multiple transitions are \emph{enabled} one could \emph{fire} non-deterministically.}
        \item{In this Petri-Net, $t_1$ is enabled but $t_2$ is not}
    \end{itemize}
    \begin{figure}
        \centering
        \intropetrinet[red][blue]{}
    \end{figure}
\end{frame}

\begin{frame}
    \frametitle{Transition Rules (continued)}
    \begin{itemize}
        \item{When a \emph{transition} fires,
            \begin{itemize}
                \item{One token is consumed from each input \emph{places}}
                \item{One token is produced in all output \emph{places}}
            \end{itemize}}
    \end{itemize}
    \begin{figure}
        \centering
        \tikzset{node distance = 0.5cm and 0.5cm}
        \intropetrinet[red][black][0.5]{}
        \intropetrinetnext[red][black][0.5]{}
    \end{figure}
\end{frame}

\begin{frame}
    \frametitle{Example: Dining Philosophers}
    \begin{figure}
        \centering
        \tikzset{node distance = 1.0cm and 1.0cm}
        \diningphilos{}
    \end{figure}
\end{frame}
\begin{frame}
    \frametitle{Example: Dining Philosophers (continued)}
    \begin{figure}
        \centering
        \tikzset{node distance = 1.0cm and 1.0cm}
        \diningphilosnext{}
    \end{figure}
\end{frame}
