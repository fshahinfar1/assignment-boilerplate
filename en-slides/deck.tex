\documentclass{beamer}
\usetheme{Warsaw}

\usepackage{graphicx}
\usepackage{amsmath}
\usepackage{amsfonts}
\usepackage{mathtools}

\usepackage{tikz}
\usetikzlibrary{petri, automata, positioning}

\usepackage{xparse}
\usepackage[
    type={CC},
    modifier={by-nc-sa},
    version={3.0},
]{doclicense}

% -----------------

\def\uniname{Sharif University of Technology}

% Add table of content at the begining of each section
\AtBeginSection[]
{
  \begin{frame}
    \frametitle{Table of Contents}
    \tableofcontents[currentsection]
  \end{frame}
}

\addtobeamertemplate{navigation symbols}{}{%
    \ifnum \value{framenumber} > 1 {
        \usebeamerfont{footline}%
        \usebeamercolor[fg]{footline}%
        % \hspace{1em}%
        % \doclicenseThis{}
        \hspace{1em}%
        \insertframenumber/\inserttotalframenumber
    } \else {} \fi
}
\setbeamercolor{footline}{fg=blue}
\setbeamerfont{footline}{series=\bfseries}

% Slide information
\title[Petri-Nets] % (optional, only for long titles)
{Syntax and Semantics of Petri-Nets}
\subtitle{Defining Petri-Nets with Transition Systems}
\author[Author, Shahinfar]
{Farbod Shahinfar} % \inst{1}
\institute[\uniname]
{
    % \inst{1}%
    Department of Computer Engineering\newline
    \uniname
}
\date[Sharif 2021] % (optional)
{Model Checking and Verification - 2021}
\subject{Computer Science}


% Project Specific Definitions
% This is the example petri-net used in the slides
\NewDocumentCommand{\intropetrinet}{O{black} O{black} O{0.6}}{
\begin{tikzpicture}[thick,scale=#3, every node/.style={scale=#3}]
\node[place, label=p1,tokens=1](p1){};
\node[transition,minimum width=2mm,minimum height=12mm,label=t1, right=of p1, color=#1] (t1) {};
\node[place, label=p5, right=of t1, tokens=2](p5){};
\node[transition,minimum width=2mm,minimum height=12mm,label=t2, right=of p5, color=#2] (t2) {};
\node[place, label=p2, above right=of t2](p2){};
\node[place, label=below:p3, below right=of t2, tokens=1](p3){};
\node[transition,minimum width=2mm,minimum height=12mm,label=t4] (t4) at ([shift=({330:2 cm})]p2){};
\node[transition,minimum width=2mm,minimum height=12mm,label=t3, below=of t4] (t3) {};
\node[place, label=p4](p4) at ([shift=(330:2 cm)]t4){};

\draw[thick]
(p1) edge[post] (t1)
(t1) edge[bend left, post] (p2)
(t1) edge[bend right, post] (p3)
(t1) edge[post] (p5)
(p5) edge [post] (t2)
(t2) edge[post] (p5)
(p2) edge[bend left, post] (t2)
(p3) edge[bend right, post] (t2)
(t4) edge[bend left, post] (p2)
(t4) edge[bend right, post] (p3)
(p3) edge[post] (t3)
(t3) edge[post] (p4)
(p4) edge[post] (t4)
;
\end{tikzpicture}
}


\NewDocumentCommand{\intropetrinetnext}{O{black} O{black} O{0.6}}{
\begin{tikzpicture}[thick,scale=#3, every node/.style={scale=#3}]
\node[place, label=p1,tokens=0](p1){};
\node[transition,minimum width=2mm,minimum height=12mm,label=t1, right=of p1, color=#1] (t1) {};
\node[place, label=p5, right=of t1, tokens=3](p5){};
\node[transition,minimum width=2mm,minimum height=12mm,label=t2, right=of p5, color=#2] (t2) {};
\node[place, label=p2, above right=of t2, tokens=1](p2){};
\node[place, label=below:p3, below right=of t2, tokens=2](p3){};
\node[transition,minimum width=2mm,minimum height=12mm,label=t4] (t4) at ([shift=({330:2 cm})]p2){};
\node[transition,minimum width=2mm,minimum height=12mm,label=t3, below=of t4] (t3) {};
\node[place, label=p4](p4) at ([shift=(330:2 cm)]t4){};

\draw[thick]
(p1) edge[post] (t1)
(t1) edge[bend left, post] (p2)
(t1) edge[bend right, post] (p3)
(t1) edge[post] (p5)
(p5) edge [post] (t2)
(t2) edge[post] (p5)
(p2) edge[bend left, post] (t2)
(p3) edge[bend right, post] (t2)
(t4) edge[bend left, post] (p2)
(t4) edge[bend right, post] (p3)
(p3) edge[post] (t3)
(t3) edge[post] (p4)
(p4) edge[post] (t4)
;
\end{tikzpicture}
}


\NewDocumentCommand{\diningphilos}{}{
\begin{tikzpicture}[thick,scale=1, every node/.style={scale=1}]

\node[place, label=H1, tokens=1](h1){};
\node[place, label=H2, tokens=1, right=of h1, xshift=2cm](h2){};
\node[place, label=H3, tokens=1, right=of h2, xshift=2cm](h3){};

\node[transition, minimum width=12mm, minimum height=2mm, label=left:G1, below=of h1](g1){};
\node[transition, minimum width=12mm, minimum height=2mm, label=left:G2, below=of h2](g2){};
\node[transition, minimum width=12mm, minimum height=2mm, label=left:G3, below=of h3](g3){};

\node[place, label=below:E1, tokens=0, below=of g1](e1){};
\node[place, label=below:E2, tokens=0, below=of g2](e2){};
\node[place, label=below:E3, tokens=0, below=of g3](e3){};

\node[place, label=F1, tokens=1, left=of e1](f1){};
\node[place, label=F2, tokens=1, right=of e1](f2){};
\node[place, label=F3, tokens=1, right=of e2](f3){};
 
\node[transition, minimum width=12mm, minimum height=2mm, label=below left:R1, tokens=0, below=of e1](r1){};
\node[transition, minimum width=12mm, minimum height=2mm, label=below left:R2, tokens=0, below=of e2](r2){};
\node[transition, minimum width=12mm, minimum height=2mm, label=below left:R3, tokens=0, below=of e3](r3){};
\draw[thick]
(h1) edge[post] (g1)
(h2) edge[post] (g2)
(h3) edge[post] (g3)
(f1) edge[post] (g1)
% (f1) edge[post] (g3)
(f2) edge[post] (g1)
(f2) edge[post] (g2)
(f3) edge[post] (g2)
(f3) edge[post] (g3)
%
(g1) edge[post, color=black] (e1)
(g2) edge[post, color=black] (e2)
(g3) edge[post, color=black] (e3)
%
(e1) edge[post] (r1)
(e2) edge[post] (r2)
(e3) edge[post] (r3)
%
% (r1) edge[post] (h1)
(r1) edge[post] (f1)
(r1) edge[post] (f2)
% (r2) edge[post] (h2)
(r2) edge[post] (f2)
(r2) edge[post] (f3)
% (r3) edge[post] (h3)
(r3) edge[post] (f3)
% (r3) edge[post] (f1)
;
\draw [->,color=blue] (f1) to ++(0,2.5) -- ++(9.5, 0) -- (g3);
\draw [->,color=red] (r1) to ++(-2.5,0) -- ++(0, 4.7) -- (h1);
\draw [->,color=red] (r2) to ++(-2.5,0) -- ++(0, 4.7) -- (h2);
\draw [->,color=red] (r3) to ++(-2.5,0) -- ++(0, 4.7) -- (h3);
\draw [->,color=blue] (r3) to ++(0,-0.8) -|  (f1);
\end{tikzpicture}
}

\NewDocumentCommand{\diningphilosnext}{}{
\begin{tikzpicture}[thick,scale=1, every node/.style={scale=1}]

\node[place, label=H1, tokens=1](h1){};
\node[place, label=H2, tokens=0, right=of h1, xshift=2cm](h2){};
\node[place, label=H3, tokens=1, right=of h2, xshift=2cm](h3){};

\node[transition, minimum width=12mm, minimum height=2mm, label=left:G1, below=of h1](g1){};
\node[transition, minimum width=12mm, minimum height=2mm, label=left:G2, below=of h2, color=green](g2){};
\node[transition, minimum width=12mm, minimum height=2mm, label=left:G3, below=of h3](g3){};

\node[place, label=below:E1, tokens=0, below=of g1](e1){};
\node[place, label=below:E2, tokens=1, below=of g2](e2){};
\node[place, label=below:E3, tokens=0, below=of g3](e3){};

\node[place, label=F1, tokens=1, left=of e1](f1){};
\node[place, label=F2, tokens=0, right=of e1](f2){};
\node[place, label=F3, tokens=0, right=of e2](f3){};
 
\node[transition, minimum width=12mm, minimum height=2mm, label=below left:R1, tokens=0, below=of e1](r1){};
\node[transition, minimum width=12mm, minimum height=2mm, label=below left:R2, tokens=0, below=of e2](r2){};
\node[transition, minimum width=12mm, minimum height=2mm, label=below left:R3, tokens=0, below=of e3](r3){};
\draw[thick]
(h1) edge[post] (g1)
(h2) edge[post] (g2)
(h3) edge[post] (g3)
(f1) edge[post] (g1)
% (f1) edge[post] (g3)
(f2) edge[post] (g1)
(f2) edge[post] (g2)
(f3) edge[post] (g2)
(f3) edge[post] (g3)
%
(g1) edge[post, color=black] (e1)
(g2) edge[post, color=black] (e2)
(g3) edge[post, color=black] (e3)
%
(e1) edge[post] (r1)
(e2) edge[post] (r2)
(e3) edge[post] (r3)
%
% (r1) edge[post] (h1)
(r1) edge[post] (f1)
(r1) edge[post] (f2)
% (r2) edge[post] (h2)
(r2) edge[post] (f2)
(r2) edge[post] (f3)
% (r3) edge[post] (h3)
(r3) edge[post] (f3)
% (r3) edge[post] (f1)
;
\draw [->,color=blue] (f1) to ++(0,2.5) -- ++(9.5, 0) -- (g3);
\draw [->,color=red] (r1) to ++(-2.5,0) -- ++(0, 4.7) -- (h1);
\draw [->,color=red] (r2) to ++(-2.5,0) -- ++(0, 4.7) -- (h2);
\draw [->,color=red] (r3) to ++(-2.5,0) -- ++(0, 4.7) -- (h3);
\draw [->,color=blue] (r3) to ++(0,-0.8) -|  (f1);
\end{tikzpicture}
}

% ------------------------------

\begin{document}
% Title
    \begin{frame}[plain]\titlepage\end{frame}
% ========================

\section[Introduction]{Introduction to Petri-Nets}
\begin{frame}
    \frametitle{What is a Petri-Net}
    \begin{itemize}
        \item It is a modeling language.
        \item It is a \emph{directed} \emph{bipartite} graph.
        \item Petri-Nets have two type of nodes: \emph{places} and \emph{transitions}.
        \item Arcs connect \emph{places} to \emph{transitions} and vice-versa.
        \item Places hold \emph{tokens}.
        \item \emph{Tokens} determine the state (also called marking) of the system.
    \end{itemize}
\end{frame}

\begin{frame}
    \frametitle{Example of a Petri-Net graph}
    \begin{itemize}
        \item{Circles are \emph{places}}
        \item{Rectangles are \emph{transitions}}
        \item{Dots are \emph{tokens}}
    \end{itemize}
    \begin{figure}
        \centering
        \intropetrinet{}
    \end{figure}
\end{frame}

\begin{frame}
    \frametitle{Transition Rules}
    \begin{itemize}
        \item{A transition is \emph{enabled} if all its input \emph{places} has at-least one token.}
        \item{If multiple transitions are \emph{enabled} one could \emph{fire} non-deterministically.}
        \item{In this Petri-Net, $t_1$ is enabled but $t_2$ is not}
    \end{itemize}
    \begin{figure}
        \centering
        \intropetrinet[red][blue]{}
    \end{figure}
\end{frame}

\begin{frame}
    \frametitle{Transition Rules (continued)}
    \begin{itemize}
        \item{When a \emph{transition} fires,
            \begin{itemize}
                \item{One token is consumed from each input \emph{places}}
                \item{One token is produced in all output \emph{places}}
            \end{itemize}}
    \end{itemize}
    \begin{figure}
        \centering
        \tikzset{node distance = 0.5cm and 0.5cm}
        \intropetrinet[red][black][0.5]{}
        \intropetrinetnext[red][black][0.5]{}
    \end{figure}
\end{frame}

\begin{frame}
    \frametitle{Example: Dining Philosophers}
    \begin{figure}
        \centering
        \tikzset{node distance = 1.0cm and 1.0cm}
        \diningphilos{}
    \end{figure}
\end{frame}
\begin{frame}
    \frametitle{Example: Dining Philosophers (continued)}
    \begin{figure}
        \centering
        \tikzset{node distance = 1.0cm and 1.0cm}
        \diningphilosnext{}
    \end{figure}
\end{frame}


\section[Structural Definition]{Structural Definition}
\begin{frame}
    \frametitle{Structural Definition of Petri-Nets}
    \begin{itemize}
        \item {Petri-Nets are defined as a tuple shown below}
    \end{itemize}
    \begin{gather*}
        C = (P, T, I, O, \mu) \\
    \end{gather*}
\end{frame}
\begin{frame}
    \frametitle{Structural Definition of Petri-Nets (continued)}
    \begin{itemize}
        \item {$P$ is the set of all possible places}
        \item {$T$ is the set of all transitions}
    \end{itemize}
    \begin{gather*}
        C = (P, T, I, O, \mu) \\
        P := \text{set of places}~~;~~
        T := \text{set of transitions} \\
    \end{gather*}
\end{frame}
\begin{frame}
    \frametitle{Structural Definition of Petri-Nets (continued)}
    \begin{itemize}
        \item {$I$ is a function returning the input places to a transition}
    \end{itemize}
    \begin{gather*}
        C = (P, T, I, O, \mu) \\
        P := \text{set of places}~~;~~
        T := \text{set of transitions} \\
        I \subset T \times \mathcal{P(P)}~~;~~
        I(t_j) := \text{set of input places of $t_j$} \\
    \end{gather*}
\end{frame}
\begin{frame}
    \frametitle{Structural Definition of Petri-Nets (continued)}
    \begin{itemize}
        \item {$O$ is a function returning the output places to a transition}
    \end{itemize}
    \begin{gather*}
        C = (P, T, I, O, \mu) \\
        P := \text{set of places}~~;~~
        T := \text{set of transitions} \\
        I \subset T \times \mathcal{P(P)}~~;~~
        I(t_j) := \text{set of input places of $t_j$} \\
        O \subset T \times \mathcal{P(P)} ~~;~~
        O(t_j) := \text{set of output places of $t_j$} \\
    \end{gather*}
\end{frame}
\begin{frame}
    \frametitle{Structural Definition of Petri-Nets (continued)}
    \begin{itemize}
        \item {$\mu$ is a function returning the number of tokens in each place}
    \end{itemize}
    \begin{gather*}
        C = (P, T, I, O, \mu) \\
        P := \text{set of places}~~;~~
        T := \text{set of transitions} \\
        I \subset T \times \mathcal{P(P)}~~;~~
        I(t_j) := \text{set of input places of $t_j$} \\
        O \subset T \times \mathcal{P(P)} ~~;~~
        O(t_j) := \text{set of output places of $t_j$} \\
        \mu \subset P \times \mathbb{N}~~;~~
        \mu(p_i) := \text{number of tokens in the $i$-th place}
    \end{gather*}
\end{frame}
\begin{frame}
    \frametitle{Structural Definition of Petri-Nets (continued)}
    \begin{gather*}
        C = (P, T, I, O, \mu) \\
        P := \text{set of places}~~;~~
        T := \text{set of transitions} \\
        I \subset T \times \mathcal{P(P)}~~;~~
        I(t_j) := \text{set of input places of $t_j$} \\
        O \subset T \times \mathcal{P(P)} ~~;~~
        O(t_j) := \text{set of output places of $t_j$} \\
        \mu \subset P \times \mathbb{N}~~;~~
        \mu(p_i) := \text{number of tokens in the $i$-th place}
    \end{gather*}
\end{frame}

\begin{frame}
    \frametitle{Transition function $\delta$}
    \begin{itemize}
        \item {Function $\delta$ is in such way that returns next marking of the system when \textbf{enable} transition $t_j$ fires.}
    \end{itemize}
    \[\mu ^ 1 = \delta(\mu^0, t_j)\]
    \begin{itemize}
        \item {An execution can be shown as a sequence of $\mu$ values.}
    \end{itemize}
    \[(\mu^0, \mu^1, \mu^2, ...)\]
\end{frame}

\begin{frame}
    \frametitle{Example 1}
    \tikzset{node distance = 0.5cm and 0.5cm}
    \begin{figure}
        \centering
        \intropetrinet[black][black][0.6]{}
    \end{figure}
    \begin{gather*}
        C = (P,T,I,O, \mu^0)
        \\
        P = \{p_1, p_2, p_3, p_4, p_5\} ~ ~ ; ~ ~T = \{t_1, t_2, t_3, t_4\}
        \\
        I = \{(t_1, \{p_1,\}), (t_2, \{p_2, p_3, p_5\}),  (t_3, \{p_3,\}),  (t_4, \{p_4,\}),\}
        \\
        O = \{(t_1, \{p_2, p_3, p_5\}), (t_2, \{p_5\}),  (t_3, \{p_4,\}),  (t_4, \{p_3, p_2,\}),\}
        \\
        \mu^0 = \{(p_1, 1), (p_2, 0)  (p_3, 1),  (p_4, 0),  (p_5, 2) \}
    \end{gather*}
\end{frame}
\begin{frame}
    \frametitle{Example 2}
    \tikzset{node distance = 0.5cm and 0.5cm}
    \begin{figure}
        \centering
        \intropetrinetnext[black][black][0.6]{}
    \end{figure}
    \begin{gather*}
        C' = (P, T, I, O, \mu^1)
        \\
        \mu^1 = \delta(\mu^0, t_1)
    \end{gather*}
\end{frame}



\section[Semantics]{Semantics of Petri-Nets: Transition Systems}
\begin{frame}
    \frametitle{Translation}
    \begin{itemize}
        \item {Petri-Nets are defined as a tuple shown below}
    \end{itemize}
    \[C = (P, T, I, O, \mu)\]
    \begin{itemize}
        \item {In the rest of this section we define Petri-Nets with transition systems defined as:}
    \end{itemize}
    \begin{gather*}
    (S, Act, \xrightarrow{a}, Init, AP, L)
    \end{gather*}
\end{frame}

\begin{frame}
    \begin{itemize}
        \item {State of the system is tuple with the size equal to the number of places in the system}
    \end{itemize}
    \begin{gather*}
        (S, Act, \xrightarrow{a}, Init, AP, L) \\
        \mathbb{W} = \{0, 1, 2, 3, ...\} ~ ~ ; ~ n = | P |
        \\
        S =\mathbb{W} ^ n
    \end{gather*}
\end{frame}

\begin{frame}
    \begin{itemize}
        \item {For each transition $t_j$ in the Petri-Net there exists
        an action $a_j$
        having $I_{a_j}$ and $O_{a_j}$.}
        \item {$I_{a_j}$ is a sequence of zeros and ones. If $p_i$ is input of $t_j$ then $I_{a_j}[i] = 1$}
        \item {$O_{a_j}$ is a sequence of zeros and ones. If $p_i$ is output of $t_j$ then $O_{a_j}[i] = 1$}
    \end{itemize}
    \begin{gather*}
        (S, Act, \xrightarrow{a}, Init, AP, L) \\
        \mathbb{W} = \{0, 1, 2, 3, ...\} ~ ~ ; ~ n = | P |
        \\
        S =\mathbb{W} ^ n
        \\
        Act = \{a_j | ~ t_j \in T\} ~ ~ ; a_j = < I_{a_j},  O_{a_j} ~ >
        \\
        I_{a_j} = < ~ val(p_i \in I(t_j)) ~ | p_i \in P>
        \\
        O_{a_j} = < ~ val(p_i \in O(t_j)) ~ | p_i \in P>
    \end{gather*}
\end{frame}

\begin{frame}
    \frametitle{Translation}
    % \begin{itemize}
    % \item {Petri-Nets are defined as a tuple shown below}
    % \end{itemize}
    \begin{gather*}
        \mathbb{W} = \{0, 1, 2, 3, ...\} ~ ~ ; n = | P |
        \\
        S =\mathbb{W} ^ n
        \\
        Act = \{a_j   | ~ t_j \in T\} ~ ~ ; a_j = < I_{a_j},  O_{a_j} ~ >
        \\
        I_{a_j} = < ~ val(p_i \in I(t_j)) ~ | p_i \in P>
        \\
        O_{a_j} = < ~ val(p_i \in O(t_j)) ~ | p_i \in P>
        \\
        Init \in S ~ ~; Init = <~ \mu^0(p_i), ...  ~ |  p_i \in P ~ >
        \\
        AP = S
        \\
        L(s_i) = {s_i,}
    \end{gather*}
\end{frame}

\begin{frame}
    \frametitle{Translation}
    \begin{itemize}
        \item {Transition semantic can be defined as below}
    \end{itemize}
    \begin{gather*}
        \xrightarrow{a_j} \ = S \times Act \times S
        \\
        {
        I_{a_j}[k] > 0 \implies s_i[k] > 0
        \over
        s_i \xrightarrow{a_j} s'_i
        }
        \\
        {
        s_i \xrightarrow{a_j} s'_i
        \over
        s'_j = ~ < ~ s_i[k] + O_{a_j}[k] - I_{a_j}[k] , ... ~ >
        }
    \end{gather*}
\end{frame}


\begin{frame}
    \frametitle{Example}
    % \begin{itemize}
    %     \item {Transition semantic can be defined as below}
    % \end{itemize}

    % \tikzset{node distance = 0.5cm and 0.5cm}
    \begin{figure}
        \centering
        \intropetrinet[black][black][0.6]{}
    \end{figure}
\end{frame}
\begin{frame}
    \frametitle{Example}
    \begin{itemize}
        \item {Transition semantic can be defined as below}
    \end{itemize}
    {\small
    \begin{gather*}
        S = \{(0,0,0,0,0), (1,0,0,0,0), ...,(1,1,1,0,0),  ..., (1,0,1,0,2), ...\}
        \\
        Act = \{a_1=<(1,0,0,0,0), (0,1,1,0,1)>, a_2=<(0, 1, 1, 0, 1), (0,0,0,0,1)>,
        \\ a_3=<(0,0,1,0,0), (0,0,0,1,0)>,  a_4=<(0,0,0,-1,0),(0,1,1,0,0)>\}
        \\
        \xrightarrow{a_j} \ = S \times Act \times S
        \\
        Init = (1,0,1,0,2)
        \\
        AP = S
        \\
        L(s_i) = {s_i}
    \end{gather*}
    }
\end{frame}

\begin{frame}
\frametitle{Example Execution}
\begin{figure}[!h]
\small
\centering
\begin{tikzpicture}[thick,scale=0.6, every node/.style={scale=0.6}]
\node[state,blue](s0){(1,0,1,0,2)};
\node[state, below left=of s0,red,xshift=-2cm](s1){(0,1,2,0,3)};
\node[state, below right=of s0,xshift=2cm](s2){(1,0,0,1,2)};
\node[state,below left=of s1](s3){(0,0,1,0,3)};
\node[state,below right=of s1 ](s4){(0,1,1,1,3)};
\node[below=of s2](emptys2){$\vdots$};
\node[below=of s3](emptys3){$\vdots$};
\node[below=of s4](emptys4){$\vdots$};
\draw[every loop,thick]
(s0) edge[bend right, auto=right] node{$a_1$} (s1)
(s0) edge[bend left, auto=left] node{$a_3$} (s2)
(s1) edge[bend right, auto=right] node{$a_2$} (s3)
(s1)edge[bend left, auto=left] node{$a_3$} (s4)
;
\end{tikzpicture}
\end{figure}
\end{frame}


% \begin{frame}
% \doclicenseThis{}
% \end{frame}
\end{document}
